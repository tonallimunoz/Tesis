\chapter{Agradecimientos}

A la Dra. Graciela Raga y el Dr. Luis Ladino del grupo de Interacción Micro y Mesoescala del Centro de Ciencias de la Atmósfera por brindarme un segundo hogar propicio para desarrollar mi investigación con todas las libertades intelectuales y las facilidades económicas y materiales. Mi estancia en su grupo me ha permitido reflexionar acerca de que la ciencia es mucho más que academia y producción, también es comunidad, autocrítica y sensibilidad. Sin el apoyo económico de la Dra. Graciela Raga a través del Programa de Ayudantes de Investigador del Consejo Nacional de Ciencia y Tecnología este trabajo de carácter autogestivo no hubiera podido ser. Sin la guía y los consejos del Dr. Luis Ladino tantas ideas en el aire no habrían podido aterrizar.\\

A Emily Rivera y David \textit{Tiger} León por enseñarme que los conocimientos científicos no pertenecen más que a la propia naturaleza pero que las ideas críticas con reconocimiento a sus autores son la base permanente que nos permite tratar de explicar nuestro universo desde nuestro particular campo de acción. Siéntanse orgullosos y vean hasta dónde ha llegado su idea que nació en una tarde cualquiera de hace cuatro años.\\

A mi director de tesis, el Dr. Héctor Solano del Consorcio para el Estudio de Zonas Metropolitanas por haber creído en el valor de mi propuesta y brindarme en el acto, total libertad creativa en el diseño de mi investigación, sin ningún tipo de interés personal más que la motivación compartida por hacer frente al problema de contaminación lumínica en mi ciudad, la Ciudad de México. Gracias por introducirme a una vastedad de pormenores teóricos y técnicos de la investigación de la luz artificial durante la noche y permitirme acceder al modelo \textit{SkyGlow} bajo el consentimiento de su autor, el Dr. Miroslav Kocifaj de la Academia Eslovaca de Ciencias, para dar respuesta a mis más viscerales preguntas, mismas que me ayudó a traducir en términos científicos rigurosos.\\

A la Reserva Ecológica del Pedregal de San Ángel por haberme brindado espacio y reflexiones en mi formación personal más allá de la educación tan rígida que recibimos en las aulas. Reconocimientos especiales se merecen Marcela Pérez, Hilda Díaz, Merly Fabila y Néstor Chavarría por generar y compartir espacios universitarios en que podemos abordar tantas cuestiones tradicionalmente reprimidas: llamar por su nombre, capitalismo, al sistema responsable de la actual crisis mundial, expresar nuestros sentimientos como parte fundamental de nuestro actuar, visibilizar nuestra identidad sexual y voltear, por fin, el rostro a esos defensores de la humanidad sepultada bajo la violencia del poder, el odio a lo diferente y la competencia por ser <<el mejor>>. Esos defensores anónimos son los colaboradores que se saltan clases para recoger basura del pedregal, las feministas que dan y dedican su vida por los derechos de las mujeres, los zapatistas que luchan por su territorio y cada uno de los migrantes e indígenas del mundo.\\

A \textit{Las Tachas}, Lilia de la Cruz, Alejandra Argüelles e Ivonne Castillo por todos los momentos compartidos en el pedregal: las noches en que alcanzamos a ver muchas estrellas dentro de la Zona Núcleo Oriente, esperamos toda la tarde a los murciélagos, medimos colas de ratones en campo, nuestros ojos brillaron al contarle a cientos de personas acerca de la magia que tiene el pedregal y, en especial, cuando su discurso en defensa del Molotito hizo brotar mis lágrimas de orgullo.\\  

A mis amigos de la Facultad de Ciencias que me sonrojaron con su presencia, me maravillaron con sus ideas, me acompañaron en la comida, respetaron sin interrogaciones mi decisión de no beber, me abrazaron en la crisis de no saber y escucharon todas las inquietudes del mundo académico, filosófico y del amor que me dejaban sin dormir. Las buenas calificaciones no sirven más que para dedicárselas a cada uno de ustedes por la paciencia que tuvieron al explicarme lo que se me dificultaba y por la entrega que le dieron a las tareas y proyectos que elaboramos juntos. Gracias Jaime \textit{Jimbo} González, Delfina \textit{Mina} Cruz, Andrea Anguiano, Aarón Contreras, Alejandro Vega y a mis \textit{Repoixs} Andrea García, Zyanya Díaz, Diana \textit{Repoio} Morales e Iván Pineda.\\

A mis \textit{CCAmigos} por todo su apoyo profesional y por ser la razón de estar de buenas todo el día en el Centro de Ciencias de la Atmósfera: a Sandrita Porras, mi alma gemela, a Diego Cabrera por todas las carcajadas, a Carito Ramírez, Fernanda Córdoba, Montserrat Silva, Orlando Peña y a Daniel Pretelín (gracias por agradecerme en tus agradecimientos, amigo).\\

A mis amigos de siempre por ser el pilar que no ha dejado que mi crisis existenciales me derriben, ustedes me han enseñando el supremo placer de vivir. Gracias a Memo Papaya por enseñarme que el amor es incondicional, a Dianita Colín, Marianita Quiroz y Ayleen Aguilar por mostrarme el camino a mi feminidad, a Alan Benitez, Alberto Fiesco y Abraham Almaguer por compartir conmigo las grandes pasiones de nuestras vidas: la teología, la música y la fotografía, respectivamente.\\

Al erario de México y a la Universidad Nacional Autónoma de México 

Aquí agregaré agradecimientos a familia (núcleo y Xalapa), Fabi, Alejandro Zenteno, profesores (Marco Antonio Miramontes, Diego Alfaro y Fernando García), programa PAECI, erario México, UNAM, Ivonne del CCA, investigadores de Barcelona (UPC Y UB), sinodales, a la AGU y a las alcaldías por los datos facilitados de alumbrado público a través del InfoDF.