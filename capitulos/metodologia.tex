\chapter{Metodología}

\section{El modelo \textit{SkyGlow}}

Marco teórico del modelo SkyGlow

http://www.unisky.sav.sk/?lang=en&page=aplikacia&subpage=glow

Gráficas tipo all-sky de radiancia en puntos específicos de luz que llega desde todas partes del cielo (180°) o en el cenit.\\

La radiancia espectral es un problema de transferencia radiativa que se puede resolver considerando condiciones de frontera:\\

Función de emisividad de la ciudad. Relativamente constante en el tiempo. Depende de la cantidad y tipo de luminarias, la geometría (tamaño) de la ciudad y la posición del observador.\\

Características fluctuantes de la atmósfera. 
Efecto del aerosol atmosférico o turbidez atmosférica (propiedades ópticas de bulto del aerosol)
Contenido de agua líquida de las nubes que es función del tamaño de gotitas de nube y su densidad espacial así como su altitud en la atmósfera.\\

Correr el modelo sólo para 20 nm y así ahorrar tiempo de computación. Justificar por qué esto es representativo.