\chapter{Recomendaciones}
\label{chap:recomendaciones}


\begin{enumerate}[I]

\item Hacer una validación de las conclusiones obtenidas en este trabajo. Considerar que la información satelital y la modelación sirve para enfocar puntos rojos de contaminación lumínica. Aunque instrumentos como las cámaras digitales calibradas y SQMs no son capaces de obtener información espectral y direccional completa podrían utilizarse para estudios a largo plazo calibrándose y combinándose con estudios periódicos tipo \textit{all sky} sobre una amplia región del EE.

\item Llevar a cabo un refinamiento del estudio realizado con el modelo \textit{SkyGlow} al estimar la contaminación lumínica no sólo producida por la Ciudad de México sino de otras ciudades aledañas como Toluca y aglomeraciones urbanas pertenecientes a la Zona Metopolitana del Valle de México. Además hacer una intercomparación (proyecto propuesto por el M. en C. Héctor Linares de la Universidad de Barcelona) con el modelo \textit{ILLUMINA} para estimar cómo objetos masivos que bloquean o reflejan la luz como la Sierra de Guadalupe, de Santa Catarina y Ajusco-Chichinauhtzin tienen influencia en la distribución de radiancia en el cielo.

\item Incluir la dimensión de la ciencia ciudadana como una herramienta para lograr la trascendencia de la investigación de la contaminación lumínica hacia la investigación-acción participativa con el objetivo de que la colectividad abone en un entendimiento social e histórico del problema. Esto con la finalidad de hacer evidentes los supuestos y significados sesgados que se le atribuyen a la abundancia de iluminación nocturna para, finalmente, promover prácticas sustentables ciudadanas en términos de energía eléctrica. 

\item Para el caso del alumbrado público: para prevenir y mitigar la contaminación lumínica no se trata de iluminar poco sino de iluminar bien. Las emisiones de luz hacia arriba deberían ser eliminadas completamente, la potencia a la que opera el alumbrado público debería regularse conforme avanza la noche, no deberían estar encendidos espectaculares, escaparates y oficinas durante toda la noche y, para futuras actualizaciones del alumbrado público, no deberían promoverse los LEDs sino fuentes de luz con dependencia espectral en longitudes de onda larga. Y sobre todo: consultar a diseñadores de iluminación y expertos en contaminación lumínica en los temas referentes a la iluminación pública. En ese ámbito considerar que las Ciencias de la Tierra surgen como la disciplina integradora que, a través de la generación de conocimiento con un enfoque socioecosistémico, logra sembrar directrices en la construcción de la sustentabilidad.

\item En relación al Programa de Racionalidad Presupuestal de la UNAM se hace un llamado para llevar a cabo la evaluación de malas prácticas de iluminación en Ciudad Universitaria, las cuales además de generar derroche energético tienen efectos adversos en la seguridad dentro del campus y en la biodiversidad.

Al reducir la contaminación lumínica en la REPSA esta podría proponerse como \textit{Urban Night Sky Place} de la Asociación Internacional de los Cielos Oscuros (IDA por sus siglas en inglés) cuya misión es <<preservar y proteger los ambientes nocturnos y la herencia de los cielos oscuros a través de una iluminación artificial exterior de calidad>> lo que sumaría protección a la biodiversidad de Ciudad Universitaria. 

\end{enumerate}