\chapter{Conclusiones}
\label{chap:conclusiones}

\begin{enumerate}[I]

\item Se logró reproducir con éxito el modelo \textit{SkyGlow} para el caso de la Ciudad de México lo cual permitió generar los productos \textbf{Mapa CL-CDMX} que se inscribe como el primer antecedente a tomar en cuenta para la mitigación de la contaminación lumínica a través de la zonificación de la iluminación pública en la Ciudad de México, y las \textbf{gráficas tipo all sky} que permitieron comprobar que el aerosol atmosférico y la nubosidad son los principales moduladores del brillo del cielo nocturno. El modelo se configuró para reproducir prácticamente cualquier escenario atmosférico para un observador situado en cualquier punto de la ciudad.

\item Se construyó el \textbf{Inventario de Alumbrado Público de la Ciudad de México} con datos públicos que de otra manera no habrían podido estudiarse en conjunto por la marcada separación que existe entre las oficinas de transparencia de las diferentes alcaldías de la ciudad. Con estos datos no sólo se alimentó el modelo sino además se efectuaron cálculos que indican que el consumo de energía eléctrica en la Ciudad de México es responsable del 6\% de las emisiones nacionales anuales de CO$_2$.

\item Se comprobó que uno de los factores más críticos a considerar para estimar el potencial efecto en la contaminación lumínica de las fuentes de luz es su dependencia espectral. Se comprobó teóricamente que el cambio de sistema de iluminación actual de la ciudad (halogenuros metálicos) a LED podría aumentar marcadamente los patrones del brillo del cielo nocturno. 

\item En mayor o menor medida se encontró presencia de contaminación lumínica en todo el territorio de la Ciudad de México. Alcaldías como Gustavo A. Madero, Venustiano Carranza, Benito Juárez, Cuauhtémoc, Iztacalco son las más contaminadas con valores de radiancia de hasta 86 nW sr$^{-1}$ cm$^{-2}$ mientras que Milpa Alta, Cuajimalpa de Morelos, Tlalpan, Tláhuac y Magdalena Contreras registran valores mínimos de hasta 6 nW sr$^{-1}$ cm$^{-2}$.

\item Espacios naturales embebidos en la ciudad como la Reserva Ecológica del Pedregal de San Ángel de la Universidad Nacional Autónoma de México son afectados por contaminación lumínica.

\end{enumerate}