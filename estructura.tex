\chapter*{Estructura de la Introducción}

Para plantear la introducción al problema de la CL en la Ciudad de México se parte de lo particular a lo general:\\

En la \textbf{\autoref{sec:fundamentos}} se introducen los conceptos físicos de la luz y las unidades en que se mide en cualquier ámbito de estudio.\\

En la \textbf{\autoref{sec:brillocielonocturno}} se describe cómo la luz producida por fuentes naturales tiene influencia en el brillo del cielo nocturno, pero siempre sin ser de carácter contaminante. También se enlistan los principales factores atmosféricos que modulan el brillo del cielo nocturno.\\

En la \textbf{\autoref{sec:luzartificial}} se presenta la luz artificial como el factor responsable de la CL y se hace un estudio multidimensional de ella: su producción, propiedades físicas y la manera en que se utiliza para la iluminación pública.\\

En la \textbf{\autoref{sec:contaminacionluminica}} se hace un recuento de los argumentos que permiten atribuir la CL al mal uso de la luz artificial y se ofrece un panorama general de las consecuencias de la misma. También se habla de los antecedentes del estudio de la CL y el marco regulatorio que se ha desarrollado en México y el mundo para hacerle frente.\\

La \textbf{\autoref{sec:estudiodecaso}} aborda el contexto del área de estudio, la Ciudad de México, y contiene uno de los principales aportes de este trabajo: el \textbf{Inventario de Alumbrado Público de la Ciudad de México}.