\chapter{Resumen}

Las ciudades albergan a la mayoría de las poblaciones humanas a nivel mundial \citep{Pedersen2018}. La región de estudio de esta tesis es la Ciudad de México, que actualmente concentra cerca de 9 millones de habitantes \citep{INEGI2015} y está inmersa en la Zona Metropolitana del Valle de México, que cuenta con una población total de alrededor de 22 millones de habitantes \citep{OCDE2015}. Esta aglomeración implica serias consecuencias ambientales y hace a la región altamente dependiente de subsidios: con sólo 7\% de la población total de México, en la Ciudad de México se consume casi un tercio del petróleo demandado en el país y 6\% del total de la energía eléctrica \citep{SENER2013, Ezcurra2011}.\\

Como consecuencia de una dinámica centralista, en el área que abarca la Ciudad de México se cuenta con alrededor de 600 mil luminarias públicas funcionando en vías primarias y secundarias \citep{INFO2019}. Por la cantidad, distribución y características de tales puntos de luz, se puede hipotetizar que la capital del país actualmente se está enfrentando a otro problema ambiental más: la contaminación lumínica (CL).\\

La CL es cualquier efecto negativo debido a la emisión de luz artificial en intensidades, direcciones, rangos espectrales y/u horarios innecesarios \citep{AtlasREPSA}. Uno de los efectos más preocupantes es la degradación de los socioecosistemas y, más específicamente, la afectación a la población humana en temas de salud, uso sustentable de energía y ética \citep{AtlasREPSA, LibroCL, Stone2017}.\\

Aunque este tipo de contaminación comenzó a considerarse un problema desde principios de la década de 1960 \citep{LibroCL}, hasta ahora no se han realizado estudios referentes al tema en la capital del país. Esta tesis es el primer antecedente de la estimación de los niveles de CL en la Ciudad de México con el análisis de datos satelitales de radiancia y el uso del modelo teórico de distribución de la luz en la atmósfera \textit{SkyGlow} desarrollado por Kocifaj \citep{Kocifaj2007}.\\

Los resultados indican que existe CL en todo el territorio de la Ciudad de México con valores de radiancia promedio comprendidos entre 6 x 10$^{-5}$ y 8.5 x 10$^{-4}$ W sr$^{-1}$ m$^{-2}$. A través de experimentos numéricos se halló que la distribución angular de la radiancia en el cielo nocturno se puede ver drásticamente modificada por las condiciones atmosféricas tales como la concentración y tipo de aerosol y presencia de nubosidad.\\

Los principales aportes de este trabajo son la creación del \textbf{Inventario de Alumbrado Público de la Ciudad de México} y el \textbf{Mapa Teórico de Contaminación Lumínica de la Ciudad de México} que se inscribe como una de las bases para llevar a cabo estudios de alta sensibilidad de los efectos de la contaminación lumínica sobre la biodiversidad de la ciudad.\\

Por último, se proponen los ejes temáticos para lograr la sustentabilidad en el uso de energía eléctrica (racionalidad y ahorro energético) en términos de iluminación pública, y se presenta el proyecto de la promoción de un \textit{International Dark Sky Park} a las afueras de la Ciudad de México.\\

