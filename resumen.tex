\chapter{Resumen}

Las ciudades albergan a la mayoría de las poblaciones humanas a nivel mundial \citep{Zari2018}. El caso de estudio de esta tesis es la Ciudad de México, la cual actualmente concentra cerca de 9 millones de habitantes \citep{INEGI2015} y está inmersa en la Zona Metropolitana del Valle de México que tiene una población total de alrededor de 22 millones de habitantes \citep{OCDE2015}. Esta aglomeración implica serias consecuencias ambientales y hace a la región altamente dependiente de subsidios: con sólo 7\% de la población total de México, en la Ciudad de México se consume casi un tercio del petróleo demandado en el país y 6\% del total de la energía eléctrica \citep{SENER2013}.\\

Como consecuencia de una dinámica centralista, en el área que abarca la Ciudad de México se cuenta con alrededor de 600 mil luminarias públicas funcionando en vías primarias y secundarias \citep{INFO2019}; por la cantidad, distribución y características de tales puntos de luz, se puede hipotetizar que la capital del país actualmente se está enfrentando a otro problema ambiental más: la contaminación lumínica (CL).\\

La CL es cualquier efecto negativo debido a la emisión de luz artificial en intensidades, direcciones, rangos espectrales u horarios innecesarios. Algunos de los efectos más apremiantes son la degradación de los socioecosistemas y, más específicamente, afectación a la población humana en temas de salud, ética y uso sustentable de energía \citep{AtlasREPSA}, \citep{LibroCL}, \citep{Stone2017}.\\

Aunque este tipo de contaminación comenzó a considerarse un problema desde principios de la década de 1960 \citep{LibroCL}, hasta ahora no se han realizado estudios referentes al tema en ninguna ciudad del país. Esta tesis es el primer antecedente de la estimación de los niveles de contaminación lumínica en la Ciudad de México, con el uso del modelo teórico de distribución de la luz en la atmósfera \textit{SkyGlow} desarrollado por Kocifaj \citep{Kocifaj2007}.\\


Los resultados indican que...

Los principales aportes de este trabajo son la creación del \textbf{Inventario de Alumbrado Público de la Ciudad de México} y el \textbf{Mapa Teórico de Contaminación Lumínica de la Ciudad de México}.
