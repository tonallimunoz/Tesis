\chapter{Metodología}

Datos de consumo de energía eléctrica por entidad federativa por el Sistema de Información Energética de la Secretaría de Energía.\\

Datos de alumbrado pública por las alcaldías de la Ciudad de México y la Agencia de Gestión Urbana a través del Instituto de Transparencia, Acceso a la Información Pública, Protección de Datos Personales y Rendición de Cuentas de la Ciudad de México.

\section{Medición de la luz en la atmósfera nocturna}

En \textit{Caracterización de la contaminación lumínica en zonas protegidas y urbanas} (2015), tesis doctoral de Salvador Ribas de la Universidad de Barcelona, se menciona dos tipos de medición del brillo del cielo nocturno:

\begin{itemize}

    \item Extensivas (a lo largo de un territorio)
    \item Continuas (emplazamiento permanente a lo largo de una red)     
    
\end{itemize}

En ambos casos, la finalidad óptima es la elaboración de un mapa de niveles de contaminación lumínica que incluya la dimensión temporal para evaluar los cambios en el brillo del cielo nocturno.\\

La red de Contaminación Lumínica Catalana es un ejemplo de medición continua de la CL.

\subsection{Unidades de medición de la luz}

Las magnitudes son unidades astronómicas de medición permiten establecer una escala de brillo aparente de objetos celestes. Algunos ejemplos (de menos a más brillantes)son:

\begin{itemize}

    \item 0.03. Vega, la estrella de referencia
    \item 6. Estrellas perceptibles a simple vista
    \item -12.6. Luna llena
    \item -26.73 Sol
    
 \end{itemize}
 
 Para la medición del brillo del cielo nocturno se utilizan diferentes unidades según el enfoque del estudio. En el caso de estudios generales (sin dependencia espectral) se retoman las magnitudes astronómicas por unidad de área.\\
 
En este sentido las magnitudes/arcosegundo se definen como el brillo medido en magnitudes extendido sobre un arcosegundo cuadrado del cielo (área).\\

Hay 360 grados en un círculo, hay 60 arcominutos en un grado, hay 21,600 arcominutos en un círculo. Hay 60 arcosegundos en un arcominuto, hay 1,296,000 arcosegundos en un círculo.\\

Estas unidades de medición son logarítmicas (lo que implica que cambios pequeños en la medición reflejan cambios grandes en el brillo del cielo).\\

El uso de esta unidad de medición requiere de ciertas consideraciones para la interpretación correcta de los resultados. Como se mencionó con anterioridad, las nubes, el aerosol atmosférico,la niebla y la neblina pueden alterar las mediciones tanto de campo como las remotas.

\begin{itemize}

    \item Prevenir mediciones oscuras anómalas
    
    La medición de campo más oscura, 22 magnitudes/arcosegundo se ha realizado con el uso de un SQM.
           
    Tal límite superior puede utilizarse para evaluar la confiabilidad de una medición anómalamente oscura que puede ser provocada por efecto de la presencia de  árboles/construcciones y/o el deterioro de carcasas de protección de los equipos de medición. 
    
    
    \item Prevenir mediciones brillantes anómalas
    
    En las mediciones de campo realizadas con SQM,hay que verificar que en el campo de visión del instrumento (generalmente 22\grad) no incida directamente la luz proveniente de luminarias o de la Luna.
    
    
 \end{itemize}
 
\subsection{Métodos de medición de la luz en la atmósfera nocturna}

Métodos experimentales

Métodos teóricos

Para la estimación regional de la CL se puede optar por alguno de los métodos arriba descritos o incluso la combinación de ambos.\\


\section{Escala de niveles de contaminación lumínica}

Los documentos pioneros en la estimación regional de la CL son:

\begin{itemize}

    \item \textit{CL en la Zona Metropolitana de Barcelona}
    
    En el documento se realiza un análisis de imágenes satelitales (VIIRS, se utiliza el modelo matemático de Cinzano (Atlas Mundial de la CL) y se hace un análisis de las características de los puntos más luminosos (alumbrado público).\\
    
    Es de interés contrastar la información satelital con mediciones de campo y el análisis del foco contaminante. Las imágenes satelitales sirven para enfocar la problemática.
    
    \item \textit{CL en el Parque Natural de Collserola}
    
    Mediciones realizadas con SQM georeferenciadas (Determina la calidad del cielo)
    
     
\end{itemize}


\subsection{Normas y leyes en México y el mundo}

Ley 31/1998 Protección de la Calidad Astronómica de los Observatorios del Instituto de Astrofísica de Canarias.\\

Ley 6/2001 Ordenación ambiental del alumbrado para la protección del medio nocturno.\\

Zonificación con 4 categorías y una especial.\\

\begin{itemize}

    \item E1. Espacios que por sus características naturales o su valor astronómico especial, sólo se puede admitir un brillo mínimo.
    
    \item E2. Áreas incluidas en ámbitos territoriales que sólo admiten un brillo reducido.
    
    \item E3. Áreas incluidas en ámbitos territoriales que admiten un brillo medio
    
    \item E4. Áreas incluidad en ámbitos territoriales que admiten un brillo alto.
    
    \item Puntoa de referencia. Puntos cercanos a las áreas de valor astronómico o natural especial incluidos en E1, para los que hay que establecer una regulación específica en función de la distancia a la que se encuentren del área en cuestión.
    
    
\end{itemize}

Según el Departamento de Estudios Luminotécnicos de la ETSEIB (UPC) en \textit{Evaluación del Impacto Ambiental Lumínico en Zonas Protegidas del Área Metropolitana de Barcelona}, los aspectos importantes para controlar la Contaminación Lumínica son:

\begin{itemize}
    \item Niveles más estrictos a los permisos de las luminarias (Comité Internationale d'Eclairage)
    
    \item Límites de luminosidad en el espacio-tiempo (periodo de protección especial a partir de las 23 hrs.)
    
     \item Cambiar el uso de luz blanca (especialmente nociva) a una temperatura de color neutra (4200 K), promoviendo las cálidas (inferiores a 3000 K)
     
\end{itemize}

El programa Starlight ¿Podría la REPSA convertirse en un lugar protegido?   

\section{Alumbrado público}

\subsection{Características físicas de diferentes focos}
\subsection{Función de emisividad urbana}

\section{El modelo \textit{SkyGlow}}

Marco teórico del modelo SkyGlow

http://www.unisky.sav.sk/?lang=en&page=aplikacia&subpage=glow

Gráficas tipo all-sky de radiancia en puntos específicos de luz que llega desde todas partes del cielo (180°) o en el cenit.\\

La radiancia espectral es un problema de transferencia radiativa que se puede resolver considerando condiciones de frontera:\\

Función de emisividad de la ciudad. Relativamente constante en el tiempo. Depende de la cantidad y tipo de luminarias, la geometría (tamaño) de la ciudad y la posición del observador.\\

Características fluctuantes de la atmósfera. 
Efecto del aerosol atmosférico o turbidez atmosférica (propiedades ópticas de bulto del aerosol)
Contenido de agua líquida de las nubes que es función del tamaño de gotitas de nube y su densidad espacial así como su altitud en la atmósfera.\\

Correr el modelo sólo para 20 nm y así ahorrar tiempo de computación. Justificar por qué esto es representativo.