\chapter{Metodología}

\section{El software \textit{Radiance Light Trends}}

\section{El modelo \textit{SkyGlow}}

Marco teórico del modelo SkyGlow

%http://www.unisky.sav.sk/?lang=en&page=aplikacia&subpage=glow


In general, if we want to know how radiation will be attenuated in the
atmosphere by aerosols, gases and/or clouds we need to solve a radiation
transfer equation, which requires information on optical properties of the
gases and particulates (such as extinction coefficients, single scattering
albedo, scattering phase function, etc.). 

La radiancia espectral es un problema de transferencia radiativa que se puede resolver considerando condiciones de frontera:\\


Función de emisividad de la ciudad. Relativamente constante en el tiempo. Depende de la cantidad y tipo de luminarias, la geometría (tamaño) de la ciudad y la posición del observador.\\

Características fluctuantes de la atmósfera. 
Efecto del aerosol atmosférico o turbidez atmosférica (propiedades ópticas de bulto del aerosol)
Contenido de agua líquida de las nubes que es función del tamaño de gotitas de nube y su densidad espacial así como su altitud en la atmósfera.\\

Depending on the city size, the
distance to a hypothetical observer and the
altitude of the cloud base, a city is split into

Urban night-sky luminance and cloud type 3

many pixels to keep the computational accur-
acy at a predefined level. Every pixel can be

characterised by its geometrical position with
respect to the observer, the total spectral
radiant flux emitted to the upper hemisphere
and spectral radiance as a function of zenith
angle. These functions are provided for
discrete wavelengths, thus simulating the
weighted spectrum of all light sources
(lamps) in a given pixel. Solano2015

Inputs del modelo

Correr el modelo sólo para 20 nm y así ahorrar tiempo de computación. Justificar por qué esto es representativo.

Productos:

Gráficas tipo all-sky de radiancia en puntos específicos de luz que llega desde todas partes del cielo (180 grados) o en el cenit.\\