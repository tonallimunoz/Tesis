\chapter{Agradecimientos}

A la Dra. Graciela Raga y el Dr. Luis Ladino del grupo de Interacción Micro y Mesoescala del Centro de Ciencias de la Atmósfera por brindarme un segundo hogar, propicio para desarrollar mi investigación con todas las libertades intelectuales y las facilidades económicas y materiales. Mi estancia en su grupo me ha permitido reflexionar acerca de que la ciencia es mucho más que academia y producción, también es comunidad y sensibilidad. Sin el apoyo económico de la Dra. Graciela Raga a través del Programa de Ayudantes de Investigador del Consejo Nacional de Ciencia y Tecnología, este trabajo de carácter autogestivo no hubiera podido ser. Sin la guía y los consejos del Dr. Luis Ladino tantas ideas en el aire no habrían podido aterrizar.\\

A mi director de tesis, el Dr. Héctor Solano del Consorcio para el Estudio de Zonas Metropolitanas por creer en el valor de mi propuesta, y brindarme en el acto, total libertad creativa para el diseño de mi investigación, por introducirme a una vastedad de pormenores teóricos y técnicos de la investigación de la contaminación lumínica y permitirme acceder al modelo \textit{SkyGlow} bajo el consentimiento de su autor, el Dr. Miroslav Kocifaj de la Academia Eslovaca de Ciencias, para dar respuesta a mis más viscerales preguntas, mismas que me ayudó a traducir en términos científicos rigurosos.\\

Al Dr. Manuel García de la Generalitat de Catalunya por su hospitalario recibimiento en Barcelona y en el Grupo de Estudios Luminotécnicos de la Universidad Politécnica de Cataluña. Gracias por las enseñanzas que me permitieron claridad para diseñar el enfoque de esta investigación. Fascinantes y útiles fueron las reuniones que sostuvimos en compañía del Dr. Eduard Masana y el M. en C. Héctor Linares de la Universidad de Barcelona y del Dr. Salvador Ribas del Parque Astronómico del Montsec. La estancia de investigación que realicé bajo la dirección del Dr. Manuel García me permitió reflexionar en torno al fenómeno de mi estudio en la escala global, fue promovida por el Dr. Héctor Solano y financiada por el Programa para Actividades Especiales de Cooperación Interinstitucional de la Universidad Nacional Autónoma de México y por mi mamá, Margarita S. Otamendi.\\

A cada uno de los integrantes de mi jurado y a quienes aceptaron ser propuestos para serlo: la Dra. Silvia Torres y la Dra. Irene Cruz del Instituto de Astronomía, el Dr. Harry Álvarez y el Dr. Omar Arellano de la Facultad de Ciencias, y los Dres. Luis Ladino y Michel Grutter del Centro de Ciencias de la Atmósfera. Gracias por su fundamental apoyo en el proceso de mi titulación.\\

Al erario de México y a la Universidad Nacional Autónoma de México (UNAM). Somos más que lábaros patrios, más que una <<máxima casa de estudios>>, más que una de las <<mejores universidades>> de iberoamérica, más que un número de titulados al año. México y la UNAM son los estudiantes asesinados por el ejército en 1968, los huelguistas de 1999 que lucharon por el derecho público de la educación y cada uno de los mexicanos que pagan impuestos a un estado históricamente abusivo y corrupto con la aspiración de un futuro más libre.\\

A cada una de las oficinas de transparencia de las alcaldías y a la Agencia de Gestión Urbana perteneciente a la Secretaría de Obras y Servicios de la Ciudad de México  por los datos brindados oportunamente a través del Instituto de Transparencia, Acceso a la Información Pública, Protección de Datos Personales y Rendición de Cuentas de la Ciudad de México.\\

A la Reserva Ecológica del Pedregal de San Ángel por abonar en mi formación personal más allá de la educación en las aulas. Reconocimientos especiales para Marcela Pérez, Hilda Díaz, Merly Fabila y Néstor Chavarría por generar y compartir espacios universitarios en que podemos abordar tantas cuestiones tradicionalmente reprimidas: llamar por su nombre, capitalismo, al sistema responsable de la actual crisis mundial, expresar nuestros sentimientos como parte fundamental de nuestro actuar, visibilizar nuestra identidad sexual y voltear, por fin, el rostro a esos defensores de la humanidad sepultada bajo la violencia del poder y el odio a lo diferente. Esos defensores anónimos son los colaboradores que se saltan clases para recoger basura del pedregal, las feministas que dan y dedican su vida por los derechos de las mujeres, los zapatistas que luchan por su territorio y cada uno de los migrantes e indígenas del mundo.\\

A \textit{Las Tachas}, Lilia De la Cruz, Alejandra Argüelles e Ivonne Castillo por todos los momentos compartidos en el pedregal: las noches en que alcanzamos a ver muchas estrellas dentro de la Zona Núcleo Oriente, esperamos toda la tarde a los murciélagos, medimos colas de ratones en el campo, nuestros ojos brillaron al contarle a cientos de personas acerca de la magia que tiene el pedregal y, en especial, cuando su discurso en defensa del Molotito hizo brotar mis lágrimas de orgullo.\\ 

A Emily Rivera y David \textit{Tiger} León por enseñarme que los conocimientos científicos no pertenecen más que a la propia naturaleza pero que las ideas con reconocimiento a sus autores son la base permanente que nos permite tratar de entender nuestro universo desde nuestro particular campo de acción. Espero que este esfuerzo les haga sentirse orgullosos y que vean hasta dónde ha llegado su idea que nació en una tarde cualquiera de hace cinco años.\\

A mis amigos de la Facultad de Ciencias que me sonrojaron con su amor, me maravillaron con sus ideas, me acompañaron en las comidas, me abrazaron en la crisis de no saber y escucharon todas las inquietudes del mundo académico, filosófico y del amor que me dejaban sin dormir. Las buenas calificaciones no sirven más que para dedicárselas a cada uno de ustedes por la paciencia que tuvieron al explicarme lo básico y lo avanzado de las asignaturas y por la entrega que le dieron a las tareas y proyectos que elaboramos juntos. Gracias Jaime \textit{Jimbo} González, Delfina \textit{Mina} Cruz, Andrea Anguiano, Melisa Sánchez, Aarón Contreras, Alejandro Vega, Marito Contreras; y a mis \textit{Repoixs}, Andrea García, Zyanya Díaz, Diana \textit{Repoio} Morales e Iván Pineda.\\

A los \textit{CCAmigos} por su apoyo profesional y por ser la razón de estar de buenas todo el día en el Centro de Ciencias de la Atmósfera: a mi alma gemela, Sandrita Porras, a Diego Cabrera por las reflexiones profundas de vida y las carcajadas; a Carito Ramírez, Fernanda Córdoba, Montserrat Silva, Juan Pablo Cerón, Orlando Peña y a Daniel Pretelín (gracias por agradecerme en tus agradecimientos, amigo).\\

A mis compañeros de vida con quienes he disfrutado tantos momentos. Gracias a Memo Papaya por enseñarme que el amor es incondicional; a Dianita Colín, Marianita Quiroz y Ayleen Aguilar por mostrarme el camino a mi feminidad; a Alan Benítez, Alberto Fiesco y Abraham Almaguer por compartir conmigo las grandes pasiones de nuestras vidas: la teología, la música y la fotografía, respectivamente.\\

\newpage

A mis maestros que han compartido su cosmovisión conmigo y que me han inspirado para forjar la propia: mi hermano Emmanuel Muñoz, el M. en C. Marco Miramontes, el Dr. Diego Alfaro, el Dr. Fernado García, la Fís. Ivonne San Miguel y el poeta Alejandro Zenteno.\\

A mi familia que me ha enseñado a ser divergente, responsable y ecléctico. A mis papás Margarita S. Otamendi y José Manuel M. Sánchez que han procurado con infinito cuidado y cariño cada uno de los pasos que he caminado. Gracias por la confianza que me han otorgado desde siempre para desempeñarme en libertad y tomar mis propias decisiones. A mi prima Gina San Vicente por enseñarme que la familia más que sangre es apoyo.\\

Dedico este trabajo a quien es mi equipo, mi inspiración y el amor de mi vida, Fabiola Trujano; y a la memoria de mi tío, el Dr. José Luis Muñoz, un gran científico mexicano cuya pasión trasciende en cada uno de sus estudiantes y colegas de laboratorio.\\

\vspace{40mm} 

\begin{center}

\textbf{\textit{Amanecer}}\\
\vspace{2mm} 


\textit{Ihcuac tlalixpan tlaneci,}\\
\textit{in miztli momiquilia,}\\
\textit{citlalimeh ixmimiqueh,}\\
\textit{in ilhuicac moxotlaltia.}\\

\vspace{2mm} 

\textit{Ompa huehca itzintlan tepetl,}\\
\textit{popocatoc hoxacaltzin,}\\
\textit{ompa yetoc notlahzotzin,}\\
\textit{noyolotzin, nocihuatzin.}\\

\vspace{2mm}

Recopilado por Miguel León Portilla 

\end{center}